\documentclass[12pt]{article}
\usepackage{graphicx}
\usepackage{verbatim}
\title{Homework 2}
\author{Mike Skalnik}

\newcommand{\s}[1]{\setcounter{enumi}{#1}}
\newcommand{\unit}[1]{\ensuremath{\, \mathrm{#1}}}
\newcommand{\e}[1]{\ensuremath{\times 10^{#1}}}

\begin{document}

\begin{flushright}{\large Homework 2\\ Mike Skalnik}\end{flushright}

Chapter 2

\begin{enumerate}
  \s{6}
  \item Assuming zero transmission time of the object itself, the transfer time is simply $2 \mathrm{RTT}_0 + \mathrm{RTT}_1 + \mathrm{RTT}_2 + \dots + \mathrm{RTT}_n$ since $\mathrm{RTT}_0$ is done twice. Once for the actual webpage and once the connection request.

  \item
    \begin{enumerate}
      \item There needs to be $2 \mathrm{RTT}_0$ for the html file and another $2 \mathrm{RTT}_0$ for each of the objects. This means a total of $18 \mathrm{RTT}_0 + \mathrm{RTT}_1 + \mathrm{RTT}_2 + \dots + \mathrm{RTT}_n$ time elapses.
      \item $2 \mathrm{RTT}_0$ for the initial html file, then another for the first 5 objects, and one more for the remaining objects. $6 \mathrm{RTT}_0 + \mathrm{RTT}_1 + \mathrm{RTT}_2 + \dots + \mathrm{RTT}_n$
      \item $2 \mathrm{RTT}_0$ for the initial transfer and $\mathrm{RTT}_0$ for the rest of the objects, for a total of $3 \mathrm{RTT}_0 + \mathrm{RTT}_1 + \mathrm{RTT}_2 + \dots + \mathrm{RTT}_n$
    \end{enumerate}
\end{enumerate}

Chapter 4

\begin{enumerate}
  \item If routers are prone to failure, then a VC network is optimal. With a VC network, a connection must be made first, this will ensure that the router is present before sending information to it. With a datagram network would have to set up new paths, along with update routing tables for routers that fail.

  \s{8}
  \item
    \begin{enumerate}
      \item
        \begin{tabular}{|l|r|}
          \hline
          Prefix Match & Interface \\ \hline
          11100000 00 & 0 \\
          11100000 01000000 & 1 \\
          11100000 01000001 & 2 \\
          11100001 & 2 \\
          Otherwise & 3\\ \hline
        \end{tabular}
      \item It matches on prefix to determine that the first address should go through interface 3, and the other two should go through 2.
    \end{enumerate}

  \item
    \begin{tabular}{|l|c|r|}
      \hline
      Destination Address Range & \# of Addresses & Interface \# \\ \hline
      0000 0000 through 0100 0000 & 64 & 0 \\
      0100 0001 through 0101 1111 & 30 & 1 \\
      0110 0000 through 1011 1111 & 95 & 2 \\
      1100 0000 through 1111 1111 & 63 & 3 \\ \hline
    \end{tabular}

  \s{14}
  \item 128.119.40.130 can be assigned on a subnet with the prefix 128.119.40.128/26. Four equal subnets would be: \\
    \begin{tabular}{r}
      128.119.40.64/28 \\
      128.119.40.80/28 \\
      128.119.40.96/28 \\
      128.119.40.112/28
    \end{tabular}

  \s{16}
  \item Payload per fragment is the MTU less 20 bytes for IP header, so $680 \unit{\frac{bytes}{fragment}}$, therefore the number of fragments is $\lceil\frac{2400-20}{680}\rceil = 4$. Each fragment will have an identification number of 422, be 700 bytes (except the last which will be 360 bytes), and have an MF flag of 1 (except the last will have a flag of 0).
\end{enumerate}
\end{document}


